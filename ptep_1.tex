\documentclass[10pt,letterpaper,titlepage]{report}
\usepackage[utf8x]{inputenc}
\usepackage{ucs}
\usepackage{amsmath}
\usepackage{amsfonts}
\usepackage{amssymb}
\usepackage{draftcopy}
\setcounter{tocdepth}{4}
\author{
	Phos $<$phosphos0@gmail.com$>$ \\
	Tristor $<$tristor@gmail.com$>$
}
\title{
	PTEP 1 - Site Tracker Integration Protocol \\
	Version 1
}
\begin{document}

\maketitle

\tableofcontents

\chapter{Introduction}

\section{Definitions}

\paragraph{Site}

The front end comprising the majority of manual user interaction will hereby be known as the ``Site''.

\paragraph{Tracker}

The software that handles ANNOUNCE and SCRAPE requests from BitTorrent clients will hereby be known as the ``Tracker''.

\paragraph{Resource}

A \textit{Uniform Resource Location} (URL) available on the \textit{Site} will hereby be known as a ``Resource''.

\chapter{Data Format}
\label{chap:dataformat}

All data sent and returned should be encoded using JavaScript Object Notation (JSON).

\section{Object Field Prefixes}

Many instances where data needs to be changed on the Site by the Tracker, it does not need to be fully replaced, but modified based on it's current value. This is solved by Object Field Prefixes. Fields in an object are prefixed with one of the following prefixes if they are to be updated, not replaced.

Object field prefixes are to be contained within two dollar signs, one on either side of the prefix (e.g.\ \$$+$\$).

\subsection{Addition Prefix}

The addition prefix is defined as ``\$$+$\$''.

\subsubsection{Numeric Fields}

On numeric fields, the addition prefix indicates that the value of the field should be added with the value currently within the database, and stored.

\subsubsection{Arrays and Strings}

On array or string fields, the addition prefix indicates that the value of the field should be concatenated to the end of the value currently within the database, and stored.

\section{Request and Response}

\subsection{Status Code}

The root object returned should always contain a ``status'' key, containing an error value. Error values are described in the following table.

\paragraph{Rationale for not Using HTTP Status Codes}

HTTP status codes do not provide a granular description of actions taken, data returned or errors incurred.

\begin{center}
	\begin{tabular}{| r | p{8cm} |}
		\hline
		\textbf{Error Code} & \textbf{Description} \\ \hline
			
		000 & Success. \\ \hline
		001 & Success, no changes made. \\ \hline
		002 & Success, no data returned. \\ \hline
		003 & Success, single object returned. \\ \hline
		004 & Success, multiple objects returned. \\ \hline
		100 & Invalid data format. \\ \hline
		101 & Invalid data field. \\ \hline
		102 & Invalid data type for field. \\ \hline
		103 & Invalid data length for field. \\ \hline
		200 & Unknown server error. \\ \hline
		201 & Temporary server error. Retry in several seconds. \\ \hline
		202 & Long-term server error. Retry in several minutes. \\ \hline
		203 & Permanent server error. Retry once every 15 minutes and refuse connections until request successful. \\ \hline
		9xx & Error codes 900 to 999 are reserved for custom error codes. \\ \hline
	\end{tabular}
\end{center}

\subsection{Data Retrieval}

\subsubsection{Object Data Retrieval}

Single objects returned from the site to the tracker should consist of only the returned object, contained in the value for the ``object'' key of the root object.

\subsubsection{Multiple Object Data Retrieval}

Multiple objects returned from the site should be contained within an array in the value of the ``objects'' key of the root object.

\subsection{Data Submission}

\subsubsection{Single Object Data Submission}

Single objects submitted to the site should be contained within the ``object'' key of the root object.

\subsubsection{Multiple Object Data Submission}

Multiple objects submitted to the site should be contained within an array in the ``objects'' key of the root object.

\chapter{Data Object Types}

\section{Base}

\subsection{User}
\label{subsec:data:base:user}

\begin{center}
	\begin{tabular}{| p{2cm} | p{2cm} | p{7cm} |}
		\hline
	
		\textbf{Name} & \textbf{Type} & \textbf{Description} \\ \hline
		
		id & string & A unique identifier for each user. \\ \hline
		can\_leech & boolean & Whether the user is allowed to leech or not. \\ \hline
		torrent\_pass & string & The user's unique announce key. \\ \hline
	\end{tabular}
\end{center}

\subsection{Torrent}

\begin{center}
	\begin{tabular}{| p{2cm} | p{2cm} | p{7cm} |}
		\hline
	
		\textbf{Name} & \textbf{Type} & \textbf{Description} \\ \hline
		
		id & string & A unique identifier for each user. \\ \hline
		info\_hash & string & The info hash of the torrent. \\ \hline
		snatched & uint32 & The amount of snatches. \\ \hline
	\end{tabular}
\end{center}

\subsection{Whitelist Client}

\begin{center}
	\begin{tabular}{| p{2cm} | p{2cm} | p{7cm} |}
		\hline
	
		\textbf{Name} & \textbf{Type} & \textbf{Description} \\ \hline
		
		client & string & The peer\_id the client sends. \\ \hline
	\end{tabular}
\end{center}

\chapter{Site Resources}

\section{Overview}

The Site Resources chapter contains detailed descriptions of the Resources provided by the Site.

This is version 1 of the Site Tracker Integration Protocol, so all Resources will be prefixed with ``/api/v1''.

\section{Format}

\subsection{Data sent from Tracker to Site}

\subsubsection{Information Updates}

Data sent from the Tracker to the Site will be stored in HTTP parameters respective to the HTTP method used.

\subsubsection{Model Updates}

Data designed to update a model will be sent over a HTTP POST as defined in \S\ref{chap:dataformat}.

\subsection{Data returned to Tracker from Site}

Data returned to the Tracker from the Site will be formatted as defined in \S\ref{chap:dataformat}.

\section{Resources}

\subsection{/api/v1/users}

Fetch a list of users.

\subsubsection{GET}

\paragraph{Request Parameters}

If no parameters are given, all users must be returned. \\

\begin{center}
	\begin{tabular}{| r | p{8cm} |}
		\hline
	
		\textbf{Parameter} & \textbf{Description} \\ \hline
	
		last\_updated\_after & When this parameter is passed, only users updated after the given UNIX timestamp in UTC should be returned. \\ \hline
	
		limit & When this parameter is passed, only the first \textit{value} users should be returned. No order is guaranteed.  \\ \hline
	
		skip & When this parameter is passed, the first \textit{value} users should be ignored, returning only users following the first \textit{value} users. No order is guaranteed. \\ \hline
	\end{tabular}
\end{center}

\paragraph{Data Returned}

An array of User objects (see \S\ref{subsec:data:base:user})

\subsubsection{POST}

Bulk-update users.

\end{document}